\documentclass[conference]{IEEEtran}
\IEEEoverridecommandlockouts
% The preceding line is only needed to identify funding in the first footnote. If that is unneeded, please comment it out.
\usepackage{cite}
\usepackage{amsmath,amssymb,amsfonts}
\usepackage{algorithmic}
\usepackage{graphicx}
\usepackage{textcomp}
\usepackage{xcolor}
\def\BibTeX{{\rm B\kern-.05em{\sc i\kern-.025em b}\kern-.08em
    T\kern-.1667em\lower.7ex\hbox{E}\kern-.125emX}}
\begin{document}

\title{Attack Path Prediction\\
{\footnotesize \textsuperscript{*}Note: Sub-titles are not captured in Xplore and
should not be used}
\thanks{Identify applicable funding agency here. If none, delete this.}
}

\author{\IEEEauthorblockN{1\textsuperscript{st} Given Name Surname}
\IEEEauthorblockA{\textit{dept. name of organization (of Aff.)} \\
\textit{name of organization (of Aff.)}\\
City, Country \\
email address or ORCID}

\and
\IEEEauthorblockN{6\textsuperscript{th} Given Name Surname}
\IEEEauthorblockA{\textit{dept. name of organization (of Aff.)} \\
\textit{name of organization (of Aff.)}\\
City, Country \\
email address or ORCID}
}

\maketitle

\begin{abstract}
This document is a model and instructions for \LaTeX.
This and the IEEEtran.cls file define the components of your paper [title, text, heads, etc.]. *CRITICAL: Do Not Use Symbols, Special Characters, Footnotes, 
or Math in Paper Title or Abstract.
\end{abstract}

\begin{IEEEkeywords}
IIoT security, attack path prediction, reinforcement learning
\end{IEEEkeywords}

\section{Introduction}

The industrial Internet of Things (IIoT) has revolutionized modern infrastructure by integrating physical industrial processes with computing and communication capabilities, significantly improved efficiency of production. However, the heterogeneous interconnected devices has expanded the attack surface \cite{10847994}. IIoT systems include critical physical infrastructures, such as manufacturing lines, water treatment plants, where security incidents may result not only data breaches but also in catastrophic physical damage. Moreover, cyber attacks in recent years are becoming more sophisticated \cite{lin2023security}. Adversaries usually employ complicated multi-stage attack strategies, and various techniques to invade target system and reach their goals in stealth \cite{pal2021analysis}. There is a critical need for proactive defense mechanisms capable of modeling the potential attack sequence and anticipating the strategies of attacker. 

A widely adopted method to evaluate the security of a system and give potential attack paths is penetration testing. By executing authorized adversarial simulations, security professionals can expose issues within a system, enabling the security team to formulate mitigation measures in advance, enhance preparedness, and reduce overall response latency \cite{9229752}. However, traditional penetration testing remains predominantly a manual endeavor, requiring testers to painstakingly analyze the target system and chain various exploits to demonstrate a compromise. This reliance on manual execution renders the process inherently laborious, complex, and time consuming \cite{terranova2024leveraging}. 

Contributions

\begin{enumerate}
    \item Incorporated realistic attack flows and strategies, which constrains the agent action space, thereby reducing unnecessary path exploration and enhancing the quality of path generation.
    \item 
\end{enumerate}

\section{Related Works}

Choi \textit{et al.} proposed an automatic generation method leveraging the MITRE ATT\&CK framework to construct attack sequences that reflect real world adversary behaviors. By modeling the attack process as a Hidden Markov Model (HMM), they utilized transition probabilities between tactics and emission probabilities of specific techniques to generate diverse and credible attack scenarios. This approach allows for the creation of datasets that satisfy specific user objectives by adjusting probability parameters derived from historical incident reports. Furthermore, to validate the reproducibility of these sequences, an attack sequence executor was developed to automatically drive attacks on the HAI testbed using the Purple Team ATT\&CK Automation module \cite{choi2021probabilistic}.

Terranova \textit{et al.} introduced a DRL-based framework designed to overcome the scalability and generalization limitations inherent in traditional graph-based or static RL approaches. They reformulated the problem using a Partially Observable Markov Decision Process (POMDP) with a "local view," where the agent focuses on source and target nodes rather than the entire network topology. This topology-independent formulation allows the agent to be trained across a diverse set of network configurations, significantly enhancing its ability to generalize to unseen environments. By utilizing the Microsoft CyberBattleSim environment, their method demonstrated the capability to autonomously identify critical attack paths and compromise nodes without requiring prior comprehensive knowledge of the network \cite{terranova2024leveraging}.

Zhang \textit{et al.} proposed a framework for tactics, techniques and procedures to mitigate the issues of static framework like MITRE ATT\&CK, such as imbalanced sample, ambiguous difference between techniques, and restrictive coverage in new attacks \cite{10944940}.

\section{Methodology}

This section delineates the proposed RL-AG framework, casting the attack path prediction problem as a sequential decision-making task. The following subsections describe the problem formulation using a Markov Decision Process (MDP), the neural network architecture utilized for policy parameterization, the optimization procedure employing Proximal Policy Optimization (PPO), and the specific metrics adopted for evaluation.

\subsection{Reinforcement Learning Model}

The interaction between the automated attacker (agent) and the IIoT network (environment) is structured as a Markov Decision Process (MDP). This framework captures the stochastic nature of network compromise through the tuple $(\mathcal{S}, \mathcal{A}, \mathcal{P}, \mathcal{R}, \gamma)$. The state $s_t \in S$ encapsulates the agent's current observation of the network topology and the compromise status of nodes at time step $t$3. To mirror the operational constraints of a real-world adversary, the state is often defined as a partial observation, revealing only the assets and vulnerabilities currently visible to the attacker. The action space $A$ encompasses the set of permissible adversarial maneuvers, such as scanning, exploit execution, and lateral movement techniques. The transition probability $P(s_{t+1} | s_t, a_t)$ governs the system dynamics, defining the probability of the environment shifting to state $s_{t+1}$ following the execution of action $a_t$7. This component accounts for the uncertainty inherent in exploit success rates and network configurations. The reward function $R(s_t, a_t)$ delivers scalar feedback to the agent, incentivizing the identification of critical attack paths. The reward $r_t$ is computed as:
\begin{equation}
    r_t = r_{compromise} - c_{cost}
\end{equation}

Where $r_{compromise}$ represents a positive incentive for compromising a node (often scaled by the asset's value), while $c_{cost}$ reflects the operational cost associated with the action.

To address attack path prediction task, Proximal Policy Optimization (PPO) is utilized because of its robust performance and stability, preventing drastic policy deviations that can hinder convergence in complex environments.

% \begin{figure}[htbp]
% \centerline{\includegraphics{figures/PPO_porcess.png}}
% \caption{Example of a figure caption.}
% \label{fig}
% \end{figure}

\begin{figure}[htbp]
    \centering
    \includegraphics[width=\linewidth]{figures/PPO_porcess.png}
    \caption{Attack Path Generation Process}
    \label{fig}
\end{figure}

\subsection{Action Space modeling}

To address the challenge of generalizing across diverse network topologies, a topology-independent action space is adopted utilizing a local view of the environment. Rather than selecting targets from a global list of $N$ nodes—which would necessitate a dynamic and unbound output layer—the agent interacts with a relative pair of nodes: a Source Node ($n_{src}$) representing the attacker's current foothold, and a Target Node ($n_{tgt}$) representing a discovered neighbor.

The action space $\mathcal{A}$ is defined as the union of four discrete sub-spaces, allowing the agent to perform exploitation, lateral movement, and graph traversal.

\begin{equation}
\mathcal{A} = \mathcal{A}_{local} \cup \mathcal{A}_{remote} \cup \mathcal{A}_{connect} \cup \mathcal{A}_{nav}
\end{equation}

Where Actions $\mathcal{A}_{local}$ targeting the source node itself, such as privilege escalation or local credential harvesting. Actions $\mathcal{A}_{remote}$ targeting the target node from the source, utilizing remote vulnerabilities to gain initial access. $\mathcal{A}_{connect}$ attempts to establish a connection to a port on the target node using previously discovered credentials stored in the agent's cache. And actions $\mathcal{A}_{nav}$ shift the Source-Target window across the graph, allowing the agent to traverse the topology without being bound to a fixed network size.

\subsection{Reward Function}

Learning objective of the agent is governed by a multi-objective reward function formulated to approximate a goal-agnostic adversary. Unlike sparse reward signals that only value the final objective, this function $R(s, a)$ provides dense feedback, balancing immediate gains. The reward received after executing action $a$ in state $s$ is calculated as a weighted linear combination of the action's outcome.

\begin{equation}
\begin{split}
    R(s, a) &= \sum_{n \in \mathcal{N}_{new}} K_{val} \cdot V(n) + K_{disc} \cdot |\Delta N_{disc}| \\
            &\quad + K_{cred} \cdot |\Delta C_{disc}| + K_{succ} \cdot \mathbb{I}_{success} \\
            & \ \ \ - K_{cost} \cdot C(a)
\end{split}
\end{equation}

Where $\mathcal{N}_{new}$ denotes the set of newly compromised nodes, with $V(n)$ representing the intrinsic value of node $n$. $|\Delta N_{disc}|$ and $|\Delta C_{disc}|$ represent the count of newly discovered nodes and credentials, respectively, incentivizing information gathering. $\mathbb{I}_{success}$ is an indicator function that applies a bonus $K_{succ}$ only when an action succeeds for the first time. $C(a)$ represents the operational cost of the action (e.g., detection risk or time consumption), which is subtracted to discourage inefficient or noisy behavior. The coefficients $K$ are hyperparameters tuned to prioritize control over simple discovery, ensuring the agent seeks critical paths rather than merely mapping the network.

\subsection{Policy Parameterization}

An Actor-Critic architecture is implemented in this framework. The stochastic policy $\pi_\theta(a|s)$ defines the probability of taking a specific action $a$ given the current state $s$. As the agent trains, it updates $\theta$ to increase the probability of choosing successful attacks and decrease the probability of failed or costly ones. The value function $V_\phi(s)$ are approximated via deep neural networks, parameterized by vectors $\theta$ and $\phi$, it predicts the expected total future reward starting from state $s$. This function serves as baseline to identify whether a specific action was better or worse than average.

The Actor network processes the state vector to output a probability distribution over the valid action space. A Softmax activation layer is typically employed to normalize the output probabilities for discrete actions.

The critic network processes the state vector to estimate the value function $V(s)$, predicting the expected discounted future returns from the current state.

\section{Experiment and Evaluation}

\section{Conclusion}

% \section{Introduction}
% This document is a model and instructions for \LaTeX.
% Please observe the conference page limits. 

% \section{Ease of Use}

% \subsection{Maintaining the Integrity of the Specifications}

% The IEEEtran class file is used to format your paper and style the text. All margins, 
% column widths, line spaces, and text fonts are prescribed; please do not 
% alter them. You may note peculiarities. For example, the head margin
% measures proportionately more than is customary. This measurement 
% and others are deliberate, using specifications that anticipate your paper 
% as one part of the entire proceedings, and not as an independent document. 
% Please do not revise any of the current designations.

% \section{Prepare Your Paper Before Styling}
% Before you begin to format your paper, first write and save the content as a 
% separate text file. Complete all content and organizational editing before 
% formatting. Please note sections \ref{AA}--\ref{SCM} below for more information on 
% proofreading, spelling and grammar.

% Keep your text and graphic files separate until after the text has been 
% formatted and styled. Do not number text heads---{\LaTeX} will do that 
% for you.

% \subsection{Abbreviations and Acronyms}\label{AA}
% Define abbreviations and acronyms the first time they are used in the text, 
% even after they have been defined in the abstract. Abbreviations such as 
% IEEE, SI, MKS, CGS, ac, dc, and rms do not have to be defined. Do not use 
% abbreviations in the title or heads unless they are unavoidable.

% \subsection{Units}
% \begin{itemize}
% \item Use either SI (MKS) or CGS as primary units. (SI units are encouraged.) English units may be used as secondary units (in parentheses). An exception would be the use of English units as identifiers in trade, such as ``3.5-inch disk drive''.
% \item Avoid combining SI and CGS units, such as current in amperes and magnetic field in oersteds. This often leads to confusion because equations do not balance dimensionally. If you must use mixed units, clearly state the units for each quantity that you use in an equation.
% \item Do not mix complete spellings and abbreviations of units: ``Wb/m\textsuperscript{2}'' or ``webers per square meter'', not ``webers/m\textsuperscript{2}''. Spell out units when they appear in text: ``. . . a few henries'', not ``. . . a few H''.
% \item Use a zero before decimal points: ``0.25'', not ``.25''. Use ``cm\textsuperscript{3}'', not ``cc''.)
% \end{itemize}

% \subsection{Equations}
% Number equations consecutively. To make your 
% equations more compact, you may use the solidus (~/~), the exp function, or 
% appropriate exponents. Italicize Roman symbols for quantities and variables, 
% but not Greek symbols. Use a long dash rather than a hyphen for a minus 
% sign. Punctuate equations with commas or periods when they are part of a 
% sentence, as in:
% \begin{equation}
% a+b=\gamma\label{eq}
% \end{equation}

% Be sure that the 
% symbols in your equation have been defined before or immediately following 
% the equation. Use ``\eqref{eq}'', not ``Eq.~\eqref{eq}'' or ``equation \eqref{eq}'', except at 
% the beginning of a sentence: ``Equation \eqref{eq} is . . .''

% \subsection{\LaTeX-Specific Advice}

% Please use ``soft'' (e.g., \verb|\eqref{Eq}|) cross references instead
% of ``hard'' references (e.g., \verb|(1)|). That will make it possible
% to combine sections, add equations, or change the order of figures or
% citations without having to go through the file line by line.

% Please don't use the \verb|{eqnarray}| equation environment. Use
% \verb|{align}| or \verb|{IEEEeqnarray}| instead. The \verb|{eqnarray}|
% environment leaves unsightly spaces around relation symbols.

% Please note that the \verb|{subequations}| environment in {\LaTeX}
% will increment the main equation counter even when there are no
% equation numbers displayed. If you forget that, you might write an
% article in which the equation numbers skip from (17) to (20), causing
% the copy editors to wonder if you've discovered a new method of
% counting.

% {\BibTeX} does not work by magic. It doesn't get the bibliographic
% data from thin air but from .bib files. If you use {\BibTeX} to produce a
% bibliography you must send the .bib files. 

% {\LaTeX} can't read your mind. If you assign the same label to a
% subsubsection and a table, you might find that Table I has been cross
% referenced as Table IV-B3. 

% {\LaTeX} does not have precognitive abilities. If you put a
% \verb|\label| command before the command that updates the counter it's
% supposed to be using, the label will pick up the last counter to be
% cross referenced instead. In particular, a \verb|\label| command
% should not go before the caption of a figure or a table.

% Do not use \verb|\nonumber| inside the \verb|{array}| environment. It
% will not stop equation numbers inside \verb|{array}| (there won't be
% any anyway) and it might stop a wanted equation number in the
% surrounding equation.

% \subsection{Some Common Mistakes}\label{SCM}
% \begin{itemize}
% \item The word ``data'' is plural, not singular.
% \item The subscript for the permeability of vacuum $\mu_{0}$, and other common scientific constants, is zero with subscript formatting, not a lowercase letter ``o''.
% \item In American English, commas, semicolons, periods, question and exclamation marks are located within quotation marks only when a complete thought or name is cited, such as a title or full quotation. When quotation marks are used, instead of a bold or italic typeface, to highlight a word or phrase, punctuation should appear outside of the quotation marks. A parenthetical phrase or statement at the end of a sentence is punctuated outside of the closing parenthesis (like this). (A parenthetical sentence is punctuated within the parentheses.)
% \item A graph within a graph is an ``inset'', not an ``insert''. The word alternatively is preferred to the word ``alternately'' (unless you really mean something that alternates).
% \item Do not use the word ``essentially'' to mean ``approximately'' or ``effectively''.
% \item In your paper title, if the words ``that uses'' can accurately replace the word ``using'', capitalize the ``u''; if not, keep using lower-cased.
% \item Be aware of the different meanings of the homophones ``affect'' and ``effect'', ``complement'' and ``compliment'', ``discreet'' and ``discrete'', ``principal'' and ``principle''.
% \item Do not confuse ``imply'' and ``infer''.
% \item The prefix ``non'' is not a word; it should be joined to the word it modifies, usually without a hyphen.
% \item There is no period after the ``et'' in the Latin abbreviation ``et al.''.
% \item The abbreviation ``i.e.'' means ``that is'', and the abbreviation ``e.g.'' means ``for example''.
% \end{itemize}
% An excellent style manual for science writers is \cite{b7}.

% \subsection{Authors and Affiliations}
% \textbf{The class file is designed for, but not limited to, six authors.} A 
% minimum of one author is required for all conference articles. Author names 
% should be listed starting from left to right and then moving down to the 
% next line. This is the author sequence that will be used in future citations 
% and by indexing services. Names should not be listed in columns nor group by 
% affiliation. Please keep your affiliations as succinct as possible (for 
% example, do not differentiate among departments of the same organization).

% \subsection{Identify the Headings}
% Headings, or heads, are organizational devices that guide the reader through 
% your paper. There are two types: component heads and text heads.

% Component heads identify the different components of your paper and are not 
% topically subordinate to each other. Examples include Acknowledgments and 
% References and, for these, the correct style to use is ``Heading 5''. Use 
% ``figure caption'' for your Figure captions, and ``table head'' for your 
% table title. Run-in heads, such as ``Abstract'', will require you to apply a 
% style (in this case, italic) in addition to the style provided by the drop 
% down menu to differentiate the head from the text.

% Text heads organize the topics on a relational, hierarchical basis. For 
% example, the paper title is the primary text head because all subsequent 
% material relates and elaborates on this one topic. If there are two or more 
% sub-topics, the next level head (uppercase Roman numerals) should be used 
% and, conversely, if there are not at least two sub-topics, then no subheads 
% should be introduced.

% \subsection{Figures and Tables}
% \paragraph{Positioning Figures and Tables} Place figures and tables at the top and 
% bottom of columns. Avoid placing them in the middle of columns. Large 
% figures and tables may span across both columns. Figure captions should be 
% below the figures; table heads should appear above the tables. Insert 
% figures and tables after they are cited in the text. Use the abbreviation 
% ``Fig.~\ref{fig}'', even at the beginning of a sentence.

% \begin{table}[htbp]
% \caption{Table Type Styles}
% \begin{center}
% \begin{tabular}{|c|c|c|c|}
% \hline
% \textbf{Table}&\multicolumn{3}{|c|}{\textbf{Table Column Head}} \\
% \cline{2-4} 
% \textbf{Head} & \textbf{\textit{Table column subhead}}& \textbf{\textit{Subhead}}& \textbf{\textit{Subhead}} \\
% \hline
% copy& More table copy$^{\mathrm{a}}$& &  \\
% \hline
% \multicolumn{4}{l}{$^{\mathrm{a}}$Sample of a Table footnote.}
% \end{tabular}
% \label{tab1}
% \end{center}
% \end{table}

% \begin{figure}[htbp]
% \centerline{\includegraphics{fig1.png}}
% \caption{Example of a figure caption.}
% \label{fig}
% \end{figure}

% Figure Labels: Use 8 point Times New Roman for Figure labels. Use words 
% rather than symbols or abbreviations when writing Figure axis labels to 
% avoid confusing the reader. As an example, write the quantity 
% ``Magnetization'', or ``Magnetization, M'', not just ``M''. If including 
% units in the label, present them within parentheses. Do not label axes only 
% with units. In the example, write ``Magnetization (A/m)'' or ``Magnetization 
% \{A[m(1)]\}'', not just ``A/m''. Do not label axes with a ratio of 
% quantities and units. For example, write ``Temperature (K)'', not 
% ``Temperature/K''.

% \section*{Acknowledgment}

% The preferred spelling of the word ``acknowledgment'' in America is without 
% an ``e'' after the ``g''. Avoid the stilted expression ``one of us (R. B. 
% G.) thanks $\ldots$''. Instead, try ``R. B. G. thanks$\ldots$''. Put sponsor 
% acknowledgments in the unnumbered footnote on the first page.

% \section*{References}

\bibliographystyle{IEEEtran}

\bibliography{references}

% Please number citations consecutively within brackets \cite{b1}. The 
% sentence punctuation follows the bracket \cite{b2}. Refer simply to the reference 
% number, as in \cite{b3}---do not use ``Ref. \cite{b3}'' or ``reference \cite{b3}'' except at 
% the beginning of a sentence: ``Reference \cite{b3} was the first $\ldots$''

% Number footnotes separately in superscripts. Place the actual footnote at 
% the bottom of the column in which it was cited. Do not put footnotes in the 
% abstract or reference list. Use letters for table footnotes.

% Unless there are six authors or more give all authors' names; do not use 
% ``et al.''. Papers that have not been published, even if they have been 
% submitted for publication, should be cited as ``unpublished'' \cite{b4}. Papers 
% that have been accepted for publication should be cited as ``in press'' \cite{b5}. 
% Capitalize only the first word in a paper title, except for proper nouns and 
% element symbols.

% For papers published in translation journals, please give the English 
% citation first, followed by the original foreign-language citation \cite{b6}.

% \begin{thebibliography}{00}
% \bibitem{b1} G. Eason, B. Noble, and I. N. Sneddon, ``On certain integrals of Lipschitz-Hankel type involving products of Bessel functions,'' Phil. Trans. Roy. Soc. London, vol. A247, pp. 529--551, April 1955.
% \bibitem{b2} J. Clerk Maxwell, A Treatise on Electricity and Magnetism, 3rd ed., vol. 2. Oxford: Clarendon, 1892, pp.68--73.
% \bibitem{b3} I. S. Jacobs and C. P. Bean, ``Fine particles, thin films and exchange anisotropy,'' in Magnetism, vol. III, G. T. Rado and H. Suhl, Eds. New York: Academic, 1963, pp. 271--350.
% \bibitem{b4} K. Elissa, ``Title of paper if known,'' unpublished.
% \bibitem{b5} R. Nicole, ``Title of paper with only first word capitalized,'' J. Name Stand. Abbrev., in press.
% \bibitem{b6} Y. Yorozu, M. Hirano, K. Oka, and Y. Tagawa, ``Electron spectroscopy studies on magneto-optical media and plastic substrate interface,'' IEEE Transl. J. Magn. Japan, vol. 2, pp. 740--741, August 1987 [Digests 9th Annual Conf. Magnetics Japan, p. 301, 1982].
% \bibitem{b7} M. Young, The Technical Writer's Handbook. Mill Valley, CA: University Science, 1989.
% \end{thebibliography}
\vspace{12pt}

\end{document}
